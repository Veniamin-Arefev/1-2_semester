\documentclass[a4paper,12pt,titlepage,finall]{article}

\usepackage[T1,T2A]{fontenc}     % форматы шрифтов
\usepackage[utf8x]{inputenc}     % кодировка символов, используемая в данном файле
\usepackage[russian]{babel}      % пакет русификации
\usepackage{tikz}                % для создания иллюстраций
\usepackage{pgfplots}            % для вывода графиков функций
\usepackage{geometry}		 % для настройки размера полей
\usepackage{indentfirst}         % для отступа в первом абзаце секции
\usepackage{multirow}            % для таблицы с результатами

% выбираем размер листа А4, все поля ставим по 3см
\geometry{a4paper,left=30mm,top=30mm,bottom=30mm,right=30mm}

\setcounter{secnumdepth}{0}      % отключаем нумерацию секций

\usepgfplotslibrary{fillbetween} % для изображения областей на графиках

\begin{document}
% Титульный лист
\begin{titlepage}
    \begin{center}
	{\small \sc Московский государственный университет \\имени М.~В.~Ломоносова\\
	Факультет вычислительной математики и кибернетики\\}
	\vfill
	{\Large \sc Отчет по заданию №6}\\
	~\\
	{\large \bf <<Сборка многомодульных программ. \\
	Вычисление корней уравнений и определенных интегралов.>>}\\ 
	~\\
	{\large \bf Вариант 9 / 4 / 2}
    \end{center}
    \begin{flushright}
	\vfill {Выполнил:\\
	студент 105 группы\\
	Арефьев~В.~А.\\
	~\\
	Преподаватель:\\
	Смирнов~А.~В.}
    \end{flushright}
    \begin{center}
	\vfill
	{\small Москва\\2020}
    \end{center}
\end{titlepage}

% Автоматически генерируем оглавление на отдельной странице
\tableofcontents
\newpage

\section{Постановка задачи}

Необходимо было с заданной точностью $\varepsilon$ вычислить площадь плоской фигуры, ограниченной тремя кривыми, уравнения
которых $y = f1(x)$, $y = f2(x)$ и $y = f3(x)$ были заданы вариантом задания.

Для решения этой задачи было сделано:
\begin{itemize}
\item написан комбинированный метод (хорд и касательных)  нахождения корня уравнения F(x) = 0
\item протестирован выше указанный метод
\item написан метод трапеции для вычисления определённого интеграла
\item протестирован выше указанный метод
\item написаны на ассемблере функции получения значений функций и их производный в нужной точке
\item написан Makefile для автоматической компиляции, линковки программы и удаления объектных файлов
\end{itemize}

\newpage

\section{Математическое обоснование}

По условию нам даны следующие функции:
\begin{itemize} 
\item $f1(x)=\frac{3}{(x-1)^2+1}$
\item $f2(x)=\sqrt{x+0.5}$
\item $f3(x)=e^{-x}$

Их производные соответственно равны:
\item $f1'(x)=\frac{6*(1-x)}{((x-1)^2+1)^2}$
\item $f2'(x)=\frac{1}{2}*(x+0.5)^\frac{-1}{2}$
\item $f3'(x)=-e^{-x}$

Их вторые производные соответственно равны:
\item $f1''(x)=\frac{24*(x-1)^2}{((x-1)^2+1)^3}-\frac{6}{((x-1)^2+1)^2}$
\item $f2''(x)=-\frac{1}{4}*(x+0.5)^\frac{-3}{2}$
\item $f3''(x)=e^{-x}$

\end{itemize}
Для нахождения точек перечения комбинированным методом (хорд и касательных)  на отрезке [a,b], функция $F(x)~=~f(x)~-~g(x)$ должна иметь $F'(x)$, которая должна быть монотонной, непрерывной и сохраняющей знак на [a,b].

Нарисовав эскизный график, я определил, что нам нужно вычислить точки пересечений графиков f1 и f2, f1 и f3, f2 и f3.

Получаем несколько уравнений:
\begin{itemize} 
\item $F1(x)=f1(x) - f2(x)$
\item $F2(x)=f1(x) - f3(x)$
\item $F3(x)=f2(x) - f3(x)$

Воспользуемся формулой $F'(x)~=~f'(x)~-~g'(x)$

\item $F1'(x)=f1'(x) - f2'(x) = \frac{6*(1-x)}{((x-1)^2+1)^2} -  \frac{1}{2}*(x+0.5)^\frac{-1}{2}$
\item $F2'(x)=f1'(x) - f3'(x) = \frac{6*(1-x)}{((x-1)^2+1)^2} + e^{-x}$
\item $F3'(x)=f2'(x) - f3'(x) = \frac{1}{2}*(x+0.5)^\frac{-1}{2} + e^{-x}$ 

\item $F1''(x)=f1''(x) - f2''(x) = \frac{24*(x-1)^2}{((x-1)^2+1)^3}-\frac{6}{((x-1)^2+1)^2} + \frac{1}{4}*(x+0.5)^\frac{-3}{2} $
\item $F2''(x)=f1''(x) - f3''(x) = \frac{24*(x-1)^2}{((x-1)^2+1)^3}-\frac{6}{((x-1)^2+1)^2} - e^{-x}$
\item $F3''(x)=f2''(x) - f3''(x) = - \frac{1}{4}*(x+0.5)^\frac{-3}{2} -e^{-x}$ 

\end{itemize}

Решим уравнения $F1'(x)=0$, $F2'(x)=0$, $F3'(x)=0$, $F1''(x)=0$, $F2''(x)=0$, $F3''(x)=0$. Для монотонности $F'(x)$ достаточно, чтобы $F''(x)$ не изменяла свой знак, т.е. не проходила через 0(не достаточное условие , а необходимое, т.е. полученные точки нужно будет дополнительно проверить). А для проверки знакосохранение $F'(x)$ рассмотрим все моменты, когда $F'(x)=0$, т.е. где $F'(x)$ может изменить знак.

Получили, что:
\begin{itemize} 

\item $F1'(x)$  изменяет свой знак в точках, принадлежащих интервалам [-0.290;-0.289] и [0.928;0.930].
\item $F2''(x)$  изменяет свой знак в точке, принадлежащей интервалу [1.055;1.060].
\item $F3'(x)$  всегда положительна, т.е. не изменяется свой знак.

\item $F1''(x)$  изменяет свой знак в точках, принадлежащих интервалам [0.450;0.455] и [1.567;1.568]
\item $F2''(x)$  изменяет свой знак в точках, принадлежащих интервалам [-0.265;-0.260], [0.322;0.324] и [1600;1602]
\item $F3''(x)$  всегда отрицательна, т.е. не изменяется свой знак.

\end{itemize}

На каждом из вышеперечисленных интервалов присутствует ровно одна такая точка.

Ввиду того, что функция $f2(x)$ отпеределена только на интервале [-0.5,+$\infty$], то функции $F1(x)$, $F3(x)$ тоже определены только на этом интервале.
Вышеперечисленные интервалы не должны включаться в интервал [a,b] для каждой функции $F(x)$ по требованиям используемых методов вычислений.
Исходя из эскиза графика(рис.~\ref{plot1}) и наложенных ограничений выберем для каждой функции интервал:

\begin{itemize} 

\item для $F1(x)$ [1.6; 20]
\item для $F2(x)$ [-0.25;0.3]
\item для $F3(x)$ [-0.45;4] (важное зачечание, мы не можем в качестве левой границы взять -0.5, т.к. в этой точке производная равна 0)

\end{itemize}


При использовании комбинированного метода приближение к точке идёт сразу с двух стороне~\cite{math}, т.е. когда разница между левом и правой точкой становится < $\varepsilon$ = 0.001, мы находимся на нужном интервале и можно брать любую его точку, например, $\frac{a+b}{2}$. Но из-за того, что данные точки будут использоваться для вычисления интеграла нам нужно взять $\varepsilon_1$ =  $\varepsilon^2:$

Значение $\varepsilon_2$ можно выбирать равным $\varepsilon*0.1$, т.к. благодаря правилу Рунге~\cite{method} вычисление требуемого числа разбиений интервала для достичение нужной точности происходит автоматически.

Введём обозначения для точек пересечения графиков p1 для f1 и f3, p2 для f2 и f3 и p3 для f1 и f2.

\begin{figure}[h]
\centering
\begin{tikzpicture}
\begin{axis}[ %grid=both,                % рисуем координатную сетку (если нужно)
             axis lines=middle,          % рисуем оси координат в привычном для математики месте
             restrict x to domain=-2:4,  % задаем диапазон значений переменной x
             restrict y to domain=-1:6,  % задаем диапазон значений функции y(x)
             axis equal,                 % требуем соблюдения пропорций по осям x и y
             enlargelimits,              % разрешаем при необходимости увеличивать диапазоны переменных
             legend cell align=left,     % задаем выравнивание в рамке обозначений
             scale=2,                    % задаем масштаб 2:1
             xticklabels={,,},           % убираем нумерацию с оси x
             yticklabels={,,}]           % убираем нумерацию с оси y

\addplot[black,samples=256,thick,name path=ox, forget plot] {0};
% первая функция
% параметр samples отвечает за качество прорисовки
\addplot[green,samples=256,thick,name path=A] {3/((x-1)^2+1)};
% описание первой функции
\addlegendentry{$f1=\frac{3}{(x-1)^2+1}$}

% добавим немного пустого места между описанием первой и второй функций
\addlegendimage{empty legend}\addlegendentry{}

% вторая функция
% здесь необходимо дополнительно ограничить диапазон значений переменной x
\addplot[blue,domain=-0.5:4,samples=256,thick,name path=B] {sqrt(x+0.5)};
\addlegendentry{$f2=\sqrt{x+0.5}$}

% дополнительное пустое место не требуется, так как формулы имеют небольшой размер по высоте

% третья функция
\addplot[red,samples=256,thick,name path=C] {exp(-x)};
\addlegendentry{$f3=e^{-x}$}

% Поскольку автоматическое вычисление точек пересечения кривых в TiKZ реализовать сложно,
% будем явно задавать координаты.
\addplot[dashed, name path=p1] coordinates { (-0.2033, 1.2254) (-0.2033, 0) };
\addplot[color=black] coordinates {(-0.2033, 0)} node [label={-10:{\small p1}}]{};

\addplot[dashed, name path=p2] coordinates { (0.1874, 0.8291) (0.1874, 0) };
\addplot[color=black] coordinates {(0.1874, 0)} node [label={-10:{\small p2}}]{};

\addplot[dashed, name path=p3] coordinates { (1.9561, 1.5672) (1.9561, 0) };
\addplot[color=black] coordinates {(1.9561, 0)} node [label={-10:{\small p3}}]{};

% закрашиваем фигуру
\addplot[blue!20,samples=256] fill between[of=A and C,soft clip={domain=-0.2033:0.1874}];
\addplot[blue!20,samples=256] fill between[of=A and B,soft clip={domain=0.1824:1.9561}];
\node [fill=white,draw=black, anchor=center] at (axis cs:1,2) {$S$};
\node [fill=white,draw=black, anchor=center] at (axis cs:-0.4,0.7) {$S_1$};
\node [fill=white,draw=black, anchor=center] at (axis cs:1.5,0.8) {$S_2$};
\node [fill=white,draw=black, anchor=center] at (axis cs:0.2,0.2) {$S_3$};

\addplot[purple!40,samples=256] fill between[of=B and C,soft clip={domain=-0.2033:0.1874}];
\addplot[lime!40,samples=256] fill between[of=B and C,soft clip={domain=0.1824:1.9561}];
\addplot[teal!40,samples=256] fill between[of=B and ox,soft clip={domain=-0.2033:0.1874}];
\addplot[teal!40,samples=256] fill between[of=C and ox,soft clip={domain=0.1824:1.9561}];
%\addplot[plum!20,samples=256] fill between[of=A and B,soft clip={domain=0.1824:1.9561}];
%\addplot[plum!20,samples=256] fill between[of=A and B,soft clip={domain=0.1824:1.9561}];

\end{axis}
\end{tikzpicture}
\caption{Эскиз плоской фигуры, ограниченной графиками заданных уравнений}
\label{plot1}
\end{figure}

Исходя из эскиза графика и найденных точек можно написать следующие равенства:
$$\int\limits_{p1}^{p3} f1(x)\,dx = S + S_1 + S_2 + S_3$$
$$\int\limits_{p1}^{p3} f2(x)\,dx = S_2 + S_3$$
$$\int\limits_{p1}^{p3} f3(x)\,dx = S_1 + S_3$$
$$\int\limits_{p1}^{p2} f2(x)\,dx + \int\limits_{p2}^{p3} f3(x)\,dx = S_3$$
Из данных равенств можно найти требуемую площадь по формуле:
$$S = \int\limits_{p1}^{p3} f1(x)\,dx - \int\limits_{p1}^{p3} f3(x)\,dx - \int\limits_{p1}^{p3} f2(x)\,dx  + \int\limits_{p1}^{p2} f2(x)\,dx + \int\limits_{p2}^{p3} f3(x)\,dx$$

\newpage

\section{Результаты экспериментов}

Координаты точек пересечения можно увидеть в таблице (Таблица ~\ref{table1}) и площадь полученной фигуры S равна 2.338597.

\begin{table}[h]
\centering
\begin{tabular}{|c|c|c|}
\hline
Кривые & $x$ & $y$ \\
\hline
1 и 2 &  -0.20333 & 1.56721 \\
2 и 3 &  0.187411 & 0.82910 \\
1 и 3 & -0.203334 & 1.22548 \\
\hline
\end{tabular}
\caption{Координаты точек пересечения}
\label{table1}
\end{table}


Итоговый график можно увидеть на рисунке.(рис.~\ref{plot2}).

\begin{figure}[h]
\centering
\begin{tikzpicture}
\begin{axis}[% grid=both,                % рисуем координатную сетку (если нужно)
             axis lines=middle,          % рисуем оси координат в привычном для математики месте
             restrict x to domain=-2:4,  % задаем диапазон значений переменной x
             restrict y to domain=-1:6,  % задаем диапазон значений функции y(x)
             axis equal,                 % требуем соблюдения пропорций по осям x и y
             enlargelimits,              % разрешаем при необходимости увеличивать диапазоны переменных
             legend cell align=left,     % задаем выравнивание в рамке обозначений
             scale=2,                    % задаем масштаб 2:1
             xticklabels={,,},           % убираем нумерацию с оси x
             yticklabels={,,}]           % убираем нумерацию с оси y

% первая функция
% параметр samples отвечает за качество прорисовки
\addplot[green,samples=256,thick,name path=A] {3/((x-1)^2+1)};
% описание первой функции
\addlegendentry{$y=\frac{3}{(x-1)^2+1}$}

% добавим немного пустого места между описанием первой и второй функций
\addlegendimage{empty legend}\addlegendentry{}

% вторая функция
% здесь необходимо дополнительно ограничить диапазон значений переменной x
\addplot[blue,domain=-0.5:4,samples=256,thick,name path=B] {sqrt(x+0.5)};
\addlegendentry{$y=\sqrt{x+0.5}$}

% дополнительное пустое место не требуется, так как формулы имеют небольшой размер по высоте

% третья функция
\addplot[red,samples=256,thick,name path=C] {exp(-x)};
\addlegendentry{$y=e^{-x}$}

% закрашиваем фигуру
\addplot[blue!20,samples=256] fill between[of=A and B,soft clip={domain=0.1824:1.9561}];
\addplot[blue!20,samples=256] fill between[of=A and C,soft clip={domain=-0.2033:0.1924}];
\addlegendentry{$S=2.3386$}

% Поскольку автоматическое вычисление точек пересечения кривых в TiKZ реализовать сложно,
% будем явно задавать координаты.
\addplot[dashed] coordinates { (-0.2033, 1.2254) (-0.2033, 0) };
\addplot[color=black] coordinates {(-0.2033, 0)} node [label={-135:{\small -0.2033}}]{};

\addplot[dashed] coordinates { (0.1874, 0.8291) (0.1874, 0) };
\addplot[color=black] coordinates {(0.1874, 0)} node [label={-45:{\small 0.1874}}]{};

\addplot[dashed] coordinates { (1.9561, 1.5672) (1.9561, 0) };
\addplot[color=black] coordinates {(1.9561, 0)} node [label={-90:{\small 1.9561}}]{};

\end{axis}
\end{tikzpicture}
\caption{Плоская фигура, ограниченная графиками заданных уравнений}
\label{plot2}
\end{figure}

\newpage

\section{Структура программы и спецификация функций}

В func.asm присутствуют функции вычисления значений функций и их производных в нужных точках. На вход подаётся x, а на выходе получается $f(x)$ или $f'(x)$
В main.c реализована вся основная логика программы.

$double~root(double~(*f)(double),~double~(*g)(double),~double~a,~double~b,~double~eps, \\ ~double~(*fdiv)(double),~double (*gdiv)(double))$ - функция, вычисляющая общую точку графиков функций $f()$ и $g()$ на интервале [a,b] c заданной точностью eps

$double~integral(double~(*f)(double),~double~a,~double~b,~double~eps)$ - функцию, вычисляющая интеграл методом трапеций с заданной точностью, использую правило Рунге

$double~nintegral(double~(*f)(double),~double~a,~double~b,~int~n)$ - функцию, вычисляющая интеграл методом трапеция разбивая [a,b] на n частей

Eдинсвенная зависимость main.c(кроме стандартных библиотек) -  это функции для вычисления значений функции и их производных в точке.

$f1(x)~f2(x)~f3(x)~f1div(x)~f2div(x)~f3div(x)$. 

Данные функции реализовы в func.asm
\newpage

\section{Сборка программы (Make-файл)}

Для сборки программы необходыми два объектных файла main.o и func.o, которые при помощи gcc линкуются. Компиляция main.o из main.c осуществляется с помощью gcc, а компиляция func.o из func.asm осуществляется с помощью nasm.
Так же присутствует инструкция cleanup, которая выполняется после успешной компиляции для удаления объектных файлов.


\newpage

\section{Отладка программы, тестирование функций}

Сначала проводилось тестирование метода нахождения точек пересечения графиков. $\varepsilon_1$=0.000001

1) $f_1(x)=x+2$ $f_1'(x)=1$  $f_2(x)=-x^2+4$ $f_2'(x)=-2*x$ 

Решая уравнение $x^2 + x - 2 = 0$ на интервале [0.1;100] можно найти корень $x=1$ Программа сделала 9 итераций.

2) $f_1(x)=x-3$ $f_1'(x)=1$  $f_2(x)=-x^3+7$ $f_2'(x)=-3*x^2$ 

Решая уравнение $x^3 + x = 10$ на интервале [0.5;5] можно найти корень $x=2$ Программа сделала 5 итераций.

3) $f_1(x)=x^2$ $f_1'(x)=2x$  $f_2(x)=-x^2+8$ $f_2'(x)=-2x$ 

Решая уравнение $2x^2 = 8$ на интервале [0.1;100] можно найти корень $x=2$. Программа сделала 9 итераций.



После проводилось тестирование метода нахождения определённого интеграла. $\varepsilon_2$=0.0001

1) $$\int\limits_{0}^{2} -\frac{x^2}{2}+4\,dx = (-\frac{x^3}{6}+4x)\bigg|_{x=0}^2 = -\frac{8}{6} + 8 = \frac{20}{3}$$
Ответ программы: "Integral = 6.66666651"  

2) $$\int\limits_{-2}^{3} -\frac{x+3}{2}+4\,dx = \frac{1+6}{2}*5 = 17.5$$
Ответ программы: "Integral = 17.50000000"  

3) $$\int\limits_{-4}^{2} \frac{x^3}{2}+2x^2+3\,dx = (\frac{x^4}{8}+\frac{2x^3}{3}+3x)\bigg|_{x=-4}^2 = 2 +\frac{2^4}{3} + 6 - (2^5 -\frac{2^7}{3} - 12) = -12 + \frac{144}{3} = 48 - 12 = 36$$
Ответ программы: "Integral = 36.00000429"  

\newpage

\section{Программа на Си и на Ассемблере}

Весь исходный код(си код и ассемблерный код), находится в приложенном архиве. Если вы это читаете, то значит, что вы ввели пароль 123.

\newpage

\section{Анализ допущенных ошибок}

Серьёзных ошибок при написании программы допущено не было, обычно компилятор сразу выявлял опечатки.

\newpage
\begin{raggedright}
\addcontentsline{toc}{section}{Список цитируемой литературы}
\begin{thebibliography}{99}
\bibitem{math} Ильин~В.\,А., Садовничий~В.\,А., Сендов~Бл.\,Х. Математический анализ. Т.\,1~---
    Москва: Наука, 1985.
\bibitem{method} Трифонов Н.П., Пильщиков В.Н. Задания практикума на ЭВМ (1 курс)  http://arch32.cs.msu.su/semestr2/\%D2\%F0\%E8\%F4\%EE\%ED\%EE\%E2\%20
\%CD.\%CF.\%2C\%20\%CF\%E8\%EB\%FC\%F9\%E8\%EA\%EE\%E2\%20\%C2.
\%CD.\%20\%C7\%E0\%E4\%E0\%ED\%E8\%FF\%20\%EF\%F0\%E0\%EA\%F2
\%E8\%EA\%F3\%EC\%E0\%20\%ED\%E0\%20\%DD\%C2\%CC\%2C2001.pdf
\end{thebibliography}
\end{raggedright}

\newpage

\end{document}
