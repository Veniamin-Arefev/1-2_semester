\documentclass[a4paper,12pt,titlepage,finall]{article}

\usepackage[T1,T2A]{fontenc}     % форматы шрифтов
\usepackage[utf8x]{inputenc}     % кодировка символов, используемая в данном файле
\usepackage[russian]{babel}      % пакет русификации
\usepackage{tikz}                % для создания иллюстраций
\usepackage{pgfplots}            % для вывода графиков функций
\usepackage{geometry}		 % для настройки размера полей
\usepackage{indentfirst}         % для отступа в первом абзаце секции
\usepackage{multirow}            % для таблицы с результатами

% выбираем размер листа А4, все поля ставим по 3см
\geometry{a4paper,left=30mm,top=30mm,bottom=30mm,right=30mm}

\setcounter{secnumdepth}{0}      % отключаем нумерацию секций

\usepgfplotslibrary{fillbetween} % для изображения областей на графиках

\begin{document}
% Титульный лист
\begin{titlepage}
    \begin{center}
	{\small \sc Московский государственный университет \\имени М.~В.~Ломоносова\\
	Факультет вычислительной математики и кибернетики\\}
	\vfill
	{\Large \sc Отчет по заданию №1}\\
	~\\
	{\large \bf <<Методы сортировки>>}\\ 
	~\\
	{\large \bf Вариант 2 / 3 / 1 / 4}
    \end{center}
    \begin{flushright}
	\vfill {Выполнил:\\
	студент 105 группы\\
	Арефьев~В.~А.\\
	~\\
	Преподаватель:\\
	Смирнов~А.~В.}
    \end{flushright}
    \begin{center}
	\vfill
	{\small Москва\\2020}
    \end{center}
\end{titlepage}

% Автоматически генерируем оглавление на отдельной странице
\tableofcontents
\newpage

\section{Постановка задачи}

Было необходимо реализовать два метода сортировки массива чисел и провести их экспериментальное сравнение. Сравнивались метод «пузырька» и рекурсивная реализация быстрой сортировки. Элементами массива были 64-разрядные целые числа. Упорядочевание проводилось  по неубыванию модулей, т.е. при сравнении элементов не учитывался знак. После написания программы было проведено некоторое количество тестов.

\newpage

\section{Результаты экспериментов}

Сложность метода «пузырька»  $O(n^2)$, т.к. максимальное количество операций равно $n^2/2$. Cложность рекурсивной реализация быстрой сортировки в худшем случае разбиения массива также равна  $O(n^2)$. Но при более удачном разбиении массива сложность последнего алгоритма равна  $O(n*log(n))$ ~\cite{cs}.
Как видно из таблиц ниже, алгоритм быстрой сортировки работает намного быстрее при больших значениях параметра n.

\begin{table}[h]
\centering
\begin{tabular}{|c|c|c|c|c|c|c|c|}
    \hline
    \multirow{2}{*}{\textbf{n}} & \multirow{2}{*}{\textbf{Параметр}} & \multicolumn{4}{|c|}{\textbf{Номер сгенерированного массива}} & \textbf{Среднее} \\
    \cline{3-6}
    & & \parbox{1.5cm}{\centering 1} & \parbox{1.5cm}{\centering 2} & \parbox{1.5cm}{\centering 3} & \parbox{1.5cm}{\centering 4} & \textbf{значение} \\
    \hline
    \multirow{2}{*}{10} & Сравнения &30&45&39&45&39,75\\
    \cline{2-7}
                        & Перемещения &14&29&15&33&22,75\\
    \hline
    \multirow{2}{*}{100} & Сравнения &4922&4760&4940&4929&4887,75\\
    \cline{2-7}
                         & Перемещения &2382&2481&2818&2422&2525,75\\
    \hline
    \multirow{2}{*}{1000} & Сравнения &498759&498905&497222&499500&498596,5\\
    \cline{2-7}
                          & Перемещения &258324&248291&254806&251516&253234,25\\
    \hline
    \multirow{2}{*}{10000} & Сравнения &49994054&49983372&49988445&49993775&49989911,5\\
    \cline{2-7}
                           & Перемещения &25056244&24614354&24825786&25052161&24887136,25\\
    \hline
\end{tabular}
\caption{Результаты работы сортировки методом «пузырька» }
\end{table}

\begin{table}[h]
\centering
\begin{tabular}{|c|c|c|c|c|c|c|c|}
    \hline
    \multirow{2}{*}{\textbf{n}} & \multirow{2}{*}{\textbf{Параметр}} & \multicolumn{4}{|c|}{\textbf{Номер сгенерированного массива}} & \textbf{Среднее} \\
    \cline{3-6}
    & & \parbox{1.5cm}{\centering 1} & \parbox{1.5cm}{\centering 2} & \parbox{1.5cm}{\centering 3} & \parbox{1.5cm}{\centering 4} & \textbf{значение} \\
    \hline
    \multirow{2}{*}{10} & Сравнения &50&58&57&59&56\\
    \cline{2-7}
                        & Перемещения &8&7&7&9&7,75\\
    \hline
    \multirow{2}{*}{100} & Сравнения &1099&1032&1016&1078&1056,25\\
    \cline{2-7}
                         & Перемещения &164&153&156&154&156,75\\
    \hline
    \multirow{2}{*}{1000} & Сравнения &16209&16094&16486&16859&16412\\
    \cline{2-7}
                          & Перемещения &2314&2325&2306&2324&2317,25\\
    \hline
    \multirow{2}{*}{10000} & Сравнения &214373&215827&220860&214810&216467,5\\
    \cline{2-7}
                           & Перемещения &30844&30653&30696&31089&30820,5\\
    \hline
\end{tabular}
\caption{Результаты работы рекурсивной реализации быстрой сортировки }
\end{table}

\newpage

\section{Структура программы и спецификация функций}
$long ~  long * generate \underline{~} array (int ~ n,time \underline{~} t ~ time \underline{~} gen)$ - генератор псевдослучайной последовательности.
На вход данной функции подаётся требуемая длина массива и ключ генератора, который основан на времени запуска программы.

$void ~ bubblesort(long ~ long ~ *array, ~ long ~ long ~ *compareout, ~ long ~ long ~ *swapout)$ - сортировка пузырьком

$void ~ quick \underline{~} sort(long ~ long ~ *array, ~ int ~ start, ~ int ~ end, ~ long ~ long ~ *compareout, ~ long ~ long ~ *swapout)$ - быстрая сортировка
На вход данным функциям подаётся сгенерированный массива и ссылки на переменные, в которые следует записать количество перемещений и сравнений.
Вторая фукнция так же принимает на вход требуемые для своей работы переменные - начальный и конечный индекс подмассива, который требуется отсортировать.

$int ~ main()$ - вся основная деятельность программы: генерирование исходной последовательности, вызов сортировок на массиве с подсчётов количества сравнений и перемещений, последующий вывод всей интересующей информации.

\newpage

\section{Отладка программы, тестирование функций}

Проверка работоспособности программы производилась сначала синтетическим путём, то есть введением строго заданного массива длина 3 с различными комбинациями цифр, после этого включался генератор случайных чисел, на котором и осуществлялась проверка программы в дальнейшем. Проверка генератора чисел проводилась путём вызова функции $generate _array$ несколько раз спустя через различные интервалы времени во время работы программы. Ошибок при тестировании кода сортировок выявленно не было.

\newpage

\section{Анализ допущенных ошибок}

При тестировании программы было выявлено, что при большой входном числе n программа выводит слишком много информации, поэтому была введена возможность выключения вывода сгенерированного массива и остортированных, путём ввода отличного от нуля числа - при желании видеть массивы и 0 - при нежелании.

\newpage
\begin{raggedright}
\addcontentsline{toc}{section}{Список цитируемой литературы}
\begin{thebibliography}{99}
\bibitem{cs} Материалы лекций | Курс АиАЯ. Лекция № 20. $http://algcourse.cs.msu.su/?page_id=30$. Дата обращение 19.02.2020. 
\end{thebibliography}
\end{raggedright}

\end{document}
